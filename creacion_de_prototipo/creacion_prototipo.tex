\documentclass[10pt]{article} 						
\usepackage[text=17cm,left=2.5cm,right=2.5cm, headsep=20pt, top=2.5cm, bottom = 2cm,letterpaper,showframe = false]{geometry} %configuración página
\usepackage{latexsym,amsmath,amssymb,amsfonts} %(símbolos de la AMS).7
\parindent = 0cm  %sangria
\usepackage[T1]{fontenc} %acentos en español
\usepackage[spanish]{babel} %español capitulos y secciones
\usepackage{graphicx} %gráficos y figuras.
%-----------------------------------------%
\usepackage{multicol}
\usepackage{titlesec}
\usepackage[rflt]{floatflt}
\usepackage{wrapfig} 
\usepackage{tikz}\usetikzlibrary{shapes.misc}
\usepackage{tikz,tkz-tab}						
\usetikzlibrary{matrix,arrows, positioning,shadows,shadings,backgrounds,
calc, shapes, tikzmark}
\usepackage{tcolorbox, empheq}
\tcbuselibrary{skins,breakable,listings,theorems}
\usepackage{xparse}							
\usepackage{pstricks}							
\usepackage[Bjornstrup]{fncychap}			
\usepackage{rotating}
\usepackage{enumerate}
\usepackage{booktabs}
\usepackage{synttree} 
\usepackage{chngcntr}
\usepackage{venndiagram}
\usepackage[all]{xy}
\usepackage{xcolor}
\usepackage{tikz}
\usetikzlibrary{datavisualization.formats.functions}
\usepackage{marginnote}										
\usepackage{fancyhdr}

%------------------------------------------
\renewcommand{\labelenumi}{\Roman{enumi}.}		%primer piso II) enumerate
\renewcommand{\labelenumii}{\arabic{enumii}$)$ }%segundo piso 2)
\renewcommand{\labelenumiii}{\alph{enumiii}$)$ }%tercer piso a)
\renewcommand{\labelenumiv}{$\bullet$}			%cuarto piso (punto)



\begin{document}
\begin{center} 
    \Huge \textbf{Creación de prototipo}\\
    \vspace{.5cm}
    \normalsize apuntes por Fode
\end{center}

\begin{flushleft}
\begin{tikzpicture}
\draw(0,1)--(16.5,1);
\end{tikzpicture}
\end{flushleft}

\section*{Prototipos de baja fidelidad e investigación de usuarios}

    \begin{itemize}
        \item La clave de un prototipo es que  sean baratos, super rápidos y pequeñas cosas que escupas rápidamente, te deshagas y recicles para seguir adelante
        \item El objetivo de los prototipos es ser vistos.\\\\
    \end{itemize}

    Debemos probar los prototipos con:

    \begin{enumerate}[\bfseries 1.]
        \item Amigos cercanos.
        \item Expertos
        \item Clientes
        \item Clientes de nuestros clientes. \\\\
    \end{enumerate}

    \begin{itemize}
        \item \textbf{La retroalimentación valiosa es cualquier retroalimentación que vincule un sentimiento del usuario a una parte especifica de la aplicación}
        \item recuerdo que debe sentirse comodo tirando el trabajo a la basura.
        \item Debemos tener claro donde queremos llegar a probar el prototipo.
        \item \textbf{Una de las preguntas más útiles para hacer cuando está construyendo un dispositivo de baja fidelidad es si un usuario puede hacer que la aplicación haga lo que se supone que debe hacer.}
        \item Debemos pensar en los puntos de interacción, es decir en cualquier lugar donde un usuario pueda presionar desplazarse o escriba para interactuar con su aplicación.
        \item \textbf{Dibujar sus prototipos simplemente en papel es la única mejor manera de trabajar en prototipos cuando comienza un proyecto desde cero y te permite cambiar de rumbo muy fácilmente} 
    \end{itemize}


\end{document}